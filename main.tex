\documentclass[12pt,a4paper,english]{article}
\usepackage{mathtools}
 
\usepackage[a4paper]{geometry}
\usepackage[utf8]{inputenc}

\usepackage{tcolorbox}

\usepackage{color}
\usepackage{tikz-cd}
\usepackage{appendix}
\usepackage{babel}
\usepackage{dsfont}
\usepackage{amsmath}
\usepackage{amssymb}
\usepackage{amsthm}
\usepackage{stmaryrd}
\usepackage{color}
\usepackage{array}
\usepackage{hyperref}
\usepackage{graphicx}
\usepackage{mathtools}
\usepackage{natbib}
\usepackage[bb=boondox]{mathalfa}
\geometry{top=3cm,bottom=3cm,left=2.5cm,right=2.5cm}
\setlength\parindent{0pt}
\renewcommand{\baselinestretch}{1.3}

\newcommand\restr[2]{{% we make the whole thing an ordinary symbol
  \left.\kern-\nulldelimiterspace % automatically resize the bar with \right
  #1 % the function
  \vphantom{\big|} % pretend it's a little taller at normal size
  \right|_{#2} % this is the delimiter
  }}
  
% definition of the "structure"
\theoremstyle{plain}
\newtheorem{thm}{Theorem}[section]
\newtheorem{lem}[thm]{Lemma}
\newtheorem{prop}[thm]{Proposition}
\newtheorem{coro}[thm]{Corollary}
\newtheorem{claim}{Claim}
\newtheorem{rmk}{Remark}

\theoremstyle{definition}
\newtheorem{conj}{Conjecture}
\newtheorem{defi}{Definition}
\newtheorem*{example}{Example}
\newtheorem{exercise}{\textbf{\textcolor{red}{Exercise}}}
\newtheorem{step}{Step}

\theoremstyle{remark}

\newtheorem*{rem}{Remark}

% define new control sequence
\newcommand{\homo}{\mathbf{Hom}}
\newcommand{\Max}{\mathbf{Max}}
\newcommand{\spec}{\mathbf{Spec}}
\newcommand{\spm}{\mathbf{Spec}_{max}}
\newcommand{\Frac}{\mathbf{Frac}}
\newcommand{\tr}{\mathrm{tr}}
\newcommand{\codim}{\mathrm{codim}}
\newcommand{\dd}{\text{d}}
\newcommand{\jac}{\textbf{Jac}}
\newcommand{\der}{\textbf{Der}}
\newcommand{\rank}{\text{rank}}
\newcommand{\sym}{\textbf{Sym}}
\newcommand{\RR}{\mathbb{R}}
\newcommand{\HH}{\mathbb{H}}
\newcommand{\ZZ}{\mathbb{Z}}
\title{Wely Law}
\date{\today}
\author{Deng Zhiyuan\footnote{Email:\ \href{mailto:dengzymathnt@outlook.com}{dengzymathnt@outlook.com}}}


\begin{document}
\maketitle
\newpage

\tableofcontents
\newpage

\section{Trace formula}

According to Proposition 1 in \cite{hoffmann1991cuspidal}, if the spectrum of Laplacian on $S^{\uparrow}=\tilde{\Gamma}\backslash\mathbb{R}\times\mathbb{H}$ is linearly bounded by a monotone function $\mu(n)$, then the corresponding values of the twist character $\chi$ are incompatible with Diophantine approximation. Hence, conversely, a parameter criterion based on the given twist character can be established. The purpose of this paper is to examine the existence of Weyl's law based on the twist character $\chi$.

\subsection{The group structure}
This section aims to analyze the space $S^{\uparrow}$, where $\tilde{\Gamma}$ is defined as: \begin{equation}\label{biggergamma}
\tilde{\Gamma}=\{(\gamma, y): \gamma\in \Gamma, y\in \phi(\gamma)\},
\end{equation} where $\Gamma$ is a Fuchsian group and $\phi$ is a group homeomorphism from $\Gamma$ to $\RR/2\pi\ZZ$. Here, $\hat{\gamma}$ is an element of the set $\tilde{\Gamma}$, consisting of pairs of $\gamma$ and $y$, where $\gamma$ is an element of $\Gamma$ and $y$ is an element of $\phi(\gamma)$.
In order to light the notation, we denote the pair $(z,t)\in \HH\times \RR $ as $\hat{z}$. And the primary focus, in the definition of $\tilde{\Gamma}$, is the group homeomorphism $\phi$, which is critical to understand. 
\begin{center}
    \begin{tikzcd}
\phi: \Gamma \arrow[rr, "\text{traslation length}"] \arrow[rrd] &  & \mathbb{R} \arrow[d]      \\
                                                                &  & \mathbb{R}/2\pi\mathbb{Z}
\end{tikzcd}
\end{center}

\textcolor{red}{the image should be defined more carefully.}


Our objective in this section is to analyze the conditions under which Weyl's law can be established with $\phi$. Furthermore, we endeavor to understand the group structure of $\tilde{\Gamma}$ and the fundamental properties of $\phi$ as essential preparation for future sections.


Before proceeding, let us review the definition of Fuchsian groups. Given $\Gamma$ as the Fuchsian group, which is a discrete subgroup of $\textbf{PSL}(2,\mathbb{R})$, 
\begin{defi}
    A subgroup of $\textbf{PSL}(2,\mathbb{R})$ acts properly discoutinuously on $\mathbb{H}$ if and only if $\Gamma$ is a Fuchsian group.
\end{defi}
Given the Fuchsian group $\Gamma$, we define the group operation of its lifted group $\Tilde{\Gamma}=\Gamma\rtimes \mathbb{R}$ as follows:
\begin{equation*}
    (\gamma_{1},y_{1})\cdot(\gamma_{2},y_{2})=(\gamma_{1}\gamma_{2},y_{1}+y_{2}-l(\gamma_{1}\gamma_{2}))
\end{equation*}
\begin{prop}
    A fundamental domain $\mathcal{F}\subset \mathbb{H}$ for Fuchsian group $\Gamma$ is a closed set such that 
    \begin{equation*}
        \Gamma\mathcal{F}:=\bigcup_{T\in\Gamma}T\mathcal{F}=\mathbb{H}
    \end{equation*}
\end{prop}

From the fundamental domain of $\Gamma$ on $\mathbb{H}$, the lift of fundamental domain is defined as 
\begin{equation*}
    \mathcal{F}^{*}=\{(\gamma\cdot z,t+\phi(\gamma))|\gamma\in \Gamma, z\in \mathcal{F}, t\in[0,\phi(\gamma)]\}
\end{equation*}

\begin{defi}
    Let $\delta\in PSL(2, \mathbb{R})$ be a hyperbolic element. Then there exists unique $\gamma \in PSL(2, \mathbb{R})$ conjugate to $\delta$, with $\gamma(z)=e^{l}z$. We call this number $l=l(\delta)$ the traslation length of $\delta$.
\end{defi}
\begin{prop}
    There is a one-to-one correspondence between the closed, oriented geodesics of the surface $S\cong \mathbb{H}/\Gamma$ and the conjugacy classes in $\Gamma$. The length of the geodesic corresponding to the conjugacy class $\{\delta\}$ is the traslation length.
\end{prop}
\begin{rmk}
As $\mathbb{R}$ is an Abelian group, the one-to-one correspondence between particular kinds of geodesics and conjugacy classes still applies to $S^{\uparrow}$.
\end{rmk}

\subsection{Trace Formula}
Before delving into the Selberg trace formula on $S^{\uparrow}$, it is necessary to define the Laplacian on $S^{\uparrow}$ and its corresponding eigenfunctions, which are denoted as follows:
\begin{align*}
     \mathcal{D}&=-y^{2}(\frac{\partial^{2}}{\partial x^{2}}+\frac{\partial^{2}}{\partial y^{2}}) + \frac{\partial^{2}}{\partial t^{2}}\\
             \mathcal{D} F(\hat{z})&= (\lambda+\mu)F(\hat{z}),
\end{align*}
 in which $F(\hat{z})=f(z)g(t)$, $\Delta f(z)=\lambda f(z)$, and $\frac{\partial^{2}}{\partial t^{2}} g(t) = \mu g(t)$.

Consider a function $G(\hat{z})\in L^{2}(S^{\uparrow})$. This function can be expanded in two ways: by using eigenfunctions of corresponding Laplacian or by using compact supported functions.  To expand $G(\hat{z})$ with a compact supported function, we choose a radially symmetric function. We define a radially symmetric function on $S^{\uparrow}$ as:
\begin{equation*}
    f(z)=\Phi[\frac{|z-z_{0}|}{\textbf{Im}(z)\textbf{Im}(z_{0})}]\cdot\Psi(|t-t_{0}|)
\end{equation*}
Then the function $G(\hat{z})$ can be written as 
\begin{equation*}
    G(\hat{z})=\sum_{\hat{\gamma}\in\Tilde{\Gamma}}\Phi[\frac{|\gamma\cdot z-z_{0}|}{\textbf{Im}(z)\textbf{Im}(z_{0})}]\cdot\Psi(|t+y -t_{0}|)
\end{equation*}

Alternatively, the function $G(\hat{z})$ can be expressed using the eigenfunctions of the Laplacian as follows: 
\begin{align*}
    G(\hat{z})&=\sum_{n=0}^{\infty} c_{n}F_{n}(\hat{z})\\
                &= \sum_{n=0}^{\infty} c_{n}f_{n}(z)g_{n}(t)
\end{align*}

Combining this two ways of expansion of the same function, the coefficients $c_{n}$ is 
\begin{align*}
    c_{n}&=\int_{\RR}\int_{\HH}\Phi[\frac{|\gamma\cdot z-z_{0}|}{\textbf{Im}(z)\textbf{Im}(z_{0})}]\cdot\Psi(|t+y -t_{0}|)F_{n}(\hat{z})\dd \mu(z)\dd t\\
    &=\int_{\RR}\Psi(|t-t_{0}|)g(t)\dd t\cdot \int_{\HH} \Phi[\frac{|\gamma\cdot z-z_{0}|}{\textbf{Im}(z)\textbf{Im}(z_{0})}] f(z)\dd \mu(z),
\end{align*}
in this integral, the value of function $\Phi$ only depends on the hyperbolic distance from $z$ to the base point $z_{0}$. 

\begin{lem}
    
\end{lem}
Then we can conclude that 
\begin{equation*}
    c_{n}=H(\lambda)W(\mu)f(z_{0})g(t_{0})
\end{equation*}

After substituting the formula for the coefficient into the expansion of function $G(\hat{z})$, the resulting expression is obtained as following:
\begin{equation*}
G(\hat{z})=\sum_{n=0}^{\infty}c_{n}
\end{equation*}
\subsection{statement}
\begin{prop}
Let $N_{\mathcal{D}}$ be the number of eigenvalues, counted with multiplicity, of the Laplacian $\mathcal{D}$ of $M$ less than $\lambda$.   
\end{prop}

\section{Ergodic point of view}

\textcolor{red}{We know that the geodesic flow has }


\newpage
\bibliographystyle{plain}
\bibliography{bib.bib}

\end{document}
